\documentclass{article}
\usepackage[utf8]{inputenc}

\usepackage{changepage}   % for the adjustwidth environment
%test
%fra lingalg
%\setlength{\textwidth}{165mm}
%\setlength{\textheight}{240mm}
\setlength{\textheight}{200mm}
\setlength{\parindent}{0mm} % S{\aa} meget rykkes ind efter afsnit
\setlength{\parskip}{\parsep}
\setlength{\headheight}{0mm}
\setlength{\headsep}{0mm}
%\setlength{\hoffset}{-2.5mm}
%\setlength{\voffset}{0mm}
%\setlength{\footskip}{15mm}
%\setlength{\oddsidemargin}{0mm}
%\setlength{\topmargin}{0mm}
%\setlength{\evensidemargin}{0mm}
%

\title{IT-Projektledelse: Ugeopgave 2}
\author{Gustav Brieghel Jensen, Bjørn Riise Brendorp, Emil Weel Sørensen  }
\date{14. September 2015}

\begin{document}

\maketitle

\section*{Spørgsmål 1}
\textit{Beskriv med egne ord, hvorfor det er nyttigt at lave en interessentanalyse, og hvad
interessentanalysens resultater bruges til igennem et it-projekt.} \\

Det er nyttigt at lave en interessentanalyse, da dette giver overblik over hvilke partier, der er involveret i projektet, og hvad deres holdning er i forhold til dette. Dette vil give udbytte i form af, at man gennem projektet kan vurdere, hvilken påvirkning ens besultninger har på hver gruppe. Et sted, dette reflekteres specielt godt, er i succeskriterierne, da man bedst muligt skal opnå succes med den bredeste gruppe.\\

Interessentanalysens resultater kan bruges gennem et projekt ved, at der kan tages højde for de forskellige gruppers interesser. Dette gøres primært i den indledende fase, da rammerne for projektets succes er fastlagt her. Efterfølgende vil det primært forårsage, at gruppen kan arbejde fokuseret mod de fastslåede mål, der opnår størst tilfredshed for flest mulige interessenter.

\newpage

\section*{Spørgsmål 2}
\textit{Identificer de 5-10 vigtigste interessenter i Tinglysningsprojektet.} \\

Her er de:
\begin{enumerate}
\item \textbf{Anvendere af systemet}\\
Herunder anvendere har vi husejere, bilejere, sælgere og ægtepagter. De har en interesse i sytemet i form af, at det vil blive nemmere for dem at komme til deres private oplysninger.
\item \textbf{Vedligeholdelse og videreudvikling}\\
Herunder de personer som skal stå for at vedligeholde og udvikle systemet. Firmaet har en interesse i at systemet blive holdt ved lige da det eller har været spild at udvikle det.
\item \textbf{Ledelsen (firmaet)}\\
Herunder den ledelse som har valgt at starte projektet. De har en interesse i at projektet bliver en succes, da de selv er kommet med forslaget og dermed ønsker at systemet bliver en succes.
\item \textbf{Ansatte (firmaet)}\\
Herunder de ansatte som er hos firmaet, og dem som står til at midste deres job hvis projektet bliver gennemført. Her er der en delt interesse for projektet, da det vil effektivisere arbejdet for dem der beholder deres arbejde, men dem som mister deres arbejde har en interesse i at det ikke lykkedes da de ikke vil være arbejdsløse.
\item \textbf{Projektgruppen}\\
Herunder projektleder og de personer som skal få projektet gennemført. De har en interesse i projektet da de bliver betalt for det, samt at de får et godt ry hvis projektet bliver en succes.

\item \textbf{Love og regler}\\
Herunder de love og regler som omhandler behandling af personfølsomme dokumenter, samt andre regler som skal påvirker projketet. Det har en interesse for klienten at projktet ikke bliver sagsøgt for brud af love og regler.

\item \textbf{Staten og kommuner} \\
Herunder staten og de kommuner som ejer ejendomme eller biler. De har en interesse i projketet da det vil gøre det nemmere for dem at holde styr på deres ejendomme og biler. 

\end{enumerate}

\newpage

\section*{Spørgsmål 3}
\textit{
Beskriv for hver af de identificerede interessenter:
\begin{enumerate}
\item Interessentens mål/interesse(r) i forhold til projektet
\item Interessentens aktivitets/interesseniveau
\item Interessentens magt
\item Hvilke konfliktpunkter kan der være ift. interessenten?
\item Hvordan håndteres/inddrages interessenten?
\end{enumerate}
}

\subsection*{Anvendere af systemet}
Denne interessentgruppe omfatter købere og sælger af ejendomme, andelsboliger og biler (slutbrugere).
\begin{enumerate}

\item Interessentens mål/interesse(r) i forhold til projektet
\begin{adjustwidth}{0.65cm}{}
   Kortere ventetid og hurtigere sagsbehandling
\end{adjustwidth}

\item Interessentens aktivitets/interesseniveau
  \begin{adjustwidth}{0.65cm}{}
 I dette tilfælde vil interessenten (anvendere af systemet) være meget aktiv hvad angår brug af systemet, da de "betjenes" igennem det. Interessentens interesseniveau vil være moderat højt, da lavere sagsbehandlingstider er at foretrække.
  \end{adjustwidth}
\item Interessentens magt
  \begin{adjustwidth}{0.65cm}{}
 Brugerne har i sidste ende meget magt, da det er dem, man vil have til at benytte sig af systemet. Vil de ikke benytte sig af systemet, har projektet været spildt arbejde, og de fordele, der var forbundet med det digitale system, bliver ikke indfriet.
  \end{adjustwidth}
\item Hvilke konfliktpunkter kan der være ift. interessenten?
  \begin{adjustwidth}{0.65cm}{}
  Ikke-digitale brugere, som ikke kan/vil benytte den digitaliserede tinglysning, og hvis sager må håndteres manuelt af en medarbejder.
  \end{adjustwidth}
\item Hvordan håndteres/inddrages interessenten?
  \begin{adjustwidth}{0.65cm}{}
  I den indledende fase kan interessenten inddragelse i forbindelse med opretning af succeskriterier. Senere kan de udføre brugertestning, så man kan forbedre og afprøve systemet.
  \end{adjustwidth}
\end{enumerate}

\end{document}
